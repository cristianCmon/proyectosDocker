% [Tamaño principal de la fuente del documento, tamaño del papel, con título (crea salto de página)]{Tipo de documento}
\documentclass[12pt, a4paper, titlepage]{article}
\usepackage[utf8]{inputenc}
% Traduce expresiones al español
\usepackage[spanish]{babel}
\usepackage{setspace}

\usepackage{graphicx}
\usepackage{geometry}
\geometry{margin=2.5cm}
\graphicspath{ {./capturas/} }

% Permite utilizar labeling para listar de forma personalizada
\usepackage{scrextend}
% Permite centrar verticalmente m{}
\usepackage{array}
% Permite colorear texto
\usepackage{xcolor}
\usepackage[dvipsnames]{xcolor}

\title{\LARGE \textbf{Práctica 5: Módulos Odoo Controlador, Herencia y Web Controllers.} \\[2ex] \Large Sistemas de Gestión Empresarial}
\author{\\[20ex]Cristian Fernández}
\date{\today}

\begin{document}
\maketitle

\doublespacing
\noindent \\
\tableofcontents % Índice

\newpage

\section{Introducción}
\noindent \\El objetivo de esta práctica es aplicar los conocimientos adquiridos sobre vistas, wizards, informes y 
controladores web. Para ello, nos ocuparemos de la implementación de cuatro actividades que abarcan diferentes contextos:

\begin{itemize}
    \item \textbf{EJ07-LigaFutbol}\textcolor{RoyalBlue}{\hspace{1em}\textit{4 puntos}}
    \begin{itemize}
        \item Se modificará el módulo para que incluya las siguientes funcionalidades:
        \begin{itemize}
            \item Reglas de puntuación especiales.\textcolor{Turquoise}{\hspace{1em}\textit{1 punto}}
            \item Botones para alterar los goles de los partidos.\textcolor{Turquoise}{\hspace{1em}\textit{1 punto}}
            \item Web controller para eliminar empates.\textcolor{Turquoise}{\hspace{1em}\textit{0,5 punto}}
            \item Informe PDF por cada partido.\textcolor{Turquoise}{\hspace{1em}\textit{0,5 punto}}
            \item Wizard para crear nuevos partidos.\textcolor{Turquoise}{\hspace{1em}\textit{0,5 punto}}
            \item Vista Graph.\textcolor{Turquoise}{\hspace{1em}\textit{0,5 punto}}
        \end{itemize}
    \end{itemize}
    \item \textbf{EJ08-API-REST\_Socios}\textcolor{RoyalBlue}{\hspace{1em}\textit{1,5 puntos}}
    \begin{itemize}
        \item Se grabará un vídeo explicativo realizando operaciones CRUD.
    \end{itemize}
    \item \textbf{Bot de Telegram}\textcolor{RoyalBlue}{\hspace{1em}\textit{3 puntos}}
    \begin{itemize}
        \item Se creará un bot de Telegram que escuchará ordenes enviadas por los usuarios.
    \end{itemize}
        \item \textbf{Generación de imágenes aleatorias con Web Controller}\textcolor{RoyalBlue}{\hspace{1em}\textit{1,5 puntos}}
    \begin{itemize}
        \item Se creará un \textit{Web Controller} que reciba parámetros, genere una imagen de píxeles aleatorios y la
        devuelva en Base64 o PNG.
    \end{itemize}
\end{itemize}

\noindent \\Los ejemplos citados anteriormente se encuentran en el siguiente repositorio.\\
\textit{https://github.com/sergarb1/OdooModulosEjemplos} \\
\newpage

\section{Actividad 01 – Modificación \textit{EJ07-LigaFutbol}}
\subsection{Reglas de puntuación especiales}
\noindent \\01
\newpage

\subsection{Botones para alterar los goles de los partidos}
\noindent \\01
\newpage

\subsection{Web Controller para eliminar empates}
\noindent \\01
\newpage

\subsection{Informe PDF por cada partido}
\noindent \\01
\newpage

\subsection{Wizard para crear nuevos partidos}
\noindent \\01
\newpage

\subsection{Vista Graph}
\noindent \\01
\newpage

\section{Actividad 02 – Pruebas \textit{EJ08-API-REST\_Socio}}
\noindent \\02
\newpage

\section{Actividad 03 – Bot Telegram API REST}
% \subsection{Informe PDF por cada partido}
% \noindent \\01
% \newpage
\noindent \\03
\newpage

\section{Actividad 04 – Generación imágenes aleatorias Web Controller}
% \subsection{Informe PDF por cada partido}
% \noindent \\01
% \newpage
\noindent \\04
\newpage

\end{document}