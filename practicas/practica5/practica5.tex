% [Tamaño principal de la fuente del documento, tamaño del papel, con título (crea salto de página)]{Tipo de documento}
\documentclass[12pt, a4paper, titlepage]{article}
\usepackage[utf8]{inputenc}
% Traduce expresiones al español
\usepackage[spanish]{babel}
\usepackage{setspace}

\usepackage{graphicx}
\usepackage{geometry}
\geometry{margin=2.5cm}
\graphicspath{ {./capturas/} }

% Permite utilizar labeling para listar de forma personalizada
\usepackage{scrextend}
% Permite centrar verticalmente m{}
\usepackage{array}
% Permite colorear texto
\usepackage{xcolor}
\usepackage[dvipsnames]{xcolor}

\title{\LARGE \textbf{Práctica 5: Módulos Odoo Controlador, Herencia y Web Controllers.} \\[2ex] \Large Sistemas de Gestión Empresarial}
\author{\\[20ex]Cristian Fernández}
\date{\today}

\begin{document}
\maketitle

\doublespacing
\noindent \\
\tableofcontents % Índice

\newpage

\section{Introducción}
\noindent \\El objetivo de esta práctica es aplicar los conocimientos adquiridos sobre vistas, wizards, informes y 
controladores web. Para ello, nos ocuparemos de la implementación de cuatro actividades que abarcan diferentes contextos:

\begin{itemize}
    \item \textbf{EJ07-LigaFutbol}\textcolor{RoyalBlue}{\hspace{1em}\textit{4 puntos}}
    \begin{itemize}
        \item Se modificará el módulo para que incluya las siguientes funcionalidades:
        \begin{itemize}
            \item Reglas de puntuación especiales.\textcolor{Turquoise}{\hspace{1em}\textit{1 punto}}
            \item Botones para alterar los goles de los partidos.\textcolor{Turquoise}{\hspace{1em}\textit{1 punto}}
            \item Web controller para eliminar empates.\textcolor{Turquoise}{\hspace{1em}\textit{0,5 punto}}
            \item Informe PDF por cada partido.\textcolor{Turquoise}{\hspace{1em}\textit{0,5 punto}}
            \item Wizard para crear nuevos partidos.\textcolor{Turquoise}{\hspace{1em}\textit{0,5 punto}}
            \item Vista Graph.\textcolor{Turquoise}{\hspace{1em}\textit{0,5 punto}}
        \end{itemize}
    \end{itemize}
    \item \textbf{EJ08-API-REST\_Socios}\textcolor{RoyalBlue}{\hspace{1em}\textit{1,5 puntos}}
    \begin{itemize}
        \item Se grabará un vídeo explicativo realizando operaciones CRUD.
    \end{itemize}
    \item \textbf{Bot de Telegram}\textcolor{RoyalBlue}{\hspace{1em}\textit{3 puntos}}
    \begin{itemize}
        \item Se creará un bot de Telegram que escuchará ordenes enviadas por los usuarios.
    \end{itemize}
    \item \textbf{Generación de imágenes aleatorias con Web Controller}\textcolor{RoyalBlue}{\hspace{1em}\textit{1,5 puntos}}
    \begin{itemize}
        \item Se creará un \textit{Web Controller} que reciba parámetros, genere una imagen de píxeles aleatorios y la
        devuelva en Base64 o PNG.
    \end{itemize}
\end{itemize}

\noindent \\Los ejemplos citados anteriormente se encuentran en el siguiente repositorio.\\
\textit{https://github.com/sergarb1/OdooModulosEjemplos} \\
\newpage

\section{Actividad 01 – Modificación \textit{EJ07-LigaFutbol}}
\subsection{Reglas de puntuación especiales}
\noindent \\Se solicita modificar la lógica de puntuación de los partidos para que, en un partido con 4 o más goles de diferencia, el
ganador se lleve 7 puntos y al perdedor se le reste 1.\\\\
Antes de comenzar será necesario modificar el código de las vistas para que el módulo se pueda instalar correctamente en Odoo.
Deberemos sustituir las etiquetas \textless tree\textgreater por \textless list\textgreater tal como se ve en la siguiente imagen:\\
\begin{figure}[hbt!]
    \centering
    \includegraphics[width=1\textwidth]{01.png}
    \caption{Aplicar estos cambios en las 3 vistas}
\end{figure}
\newpage
\noindent \\Una vez instalado el módulo accedemos a él y vemos lo siguiente:\\
\begin{figure}[hbt!]
    \centering
    \includegraphics[width=0.7\textwidth]{02.png}
    \caption{Las 4 vistas del módulo}
\end{figure}
\noindent \\\\Tras añadir los 10 primeros equipos de la primera división española nos quedará así:\\
\begin{figure}[hbt!]
    \centering
    \includegraphics[width=0.7\textwidth]{03.png}
\end{figure}
\newpage
\noindent \\
\begin{figure}[hbt!]
    \centering
    \includegraphics[width=0.7\textwidth]{04.png}
    \caption{Podemos simular varios partidos de prueba...}
\end{figure}
\noindent \\\\
\begin{figure}[hbt!]
    \centering
    \includegraphics[width=0.9\textwidth]{05.png}
    \caption{Para ver la puntuación genérica en la clasificación general}
\end{figure}
\newpage
\noindent \\Ahora echémosle un vistazo al código del módulo. Básicamente nos encontramos con 2 modelos, \textit{liga\_partido.py} y \textit{liga\_equipo.py}.
En este último se calcula la puntuación con un campo computado llamado puntos tal como se ve en la siguiente imagen:\\
\begin{figure}[hbt!]
    \centering
    \includegraphics[width=1\textwidth]{06.png}
\end{figure}
\noindent \\Dada la lógica que queremos implementar para la puntuación vamos a convertir ese campo computado en uno normal, tal que así:\\
\begin{figure}[hbt!]
    \centering
    \includegraphics[width=1\textwidth]{07.png}
\end{figure}
\noindent \\Cambiamos de modelo y nos vamos a \textit{liga\_partido.py} y añadimos el campo creado en el paso anterior dentro de la función
\textit{actualizoRegistrosEquipo()}:\\
\begin{figure}[hbt!]
    \centering
    \includegraphics[width=1\textwidth]{08.png}
\end{figure}
\newpage
\noindent \\En la misma función deberemos modificar todo este código...\\\\
\begin{figure}[hbt!]
    \centering
    \includegraphics[width=1\textwidth]{09.png}
    \caption{Código original...}
\end{figure}
\noindent \\\\\\Por este otro en el que añadiremos la nueva lógica de puntuación, tal como se muestran en las capturas de la página siguiente:
\newpage
\noindent \\\\\\
\begin{figure}[hbt!]
    \centering
    \includegraphics[width=1\textwidth]{10.png}
    \caption{Código para equipo local}
\end{figure}
\newpage
\noindent \\\\\\
\begin{figure}[hbt!]
    \centering
    \includegraphics[width=1\textwidth]{11.png}
    \caption{Código para equipo visitante}
\end{figure}
\noindent \\\\\\Es necesario matizar que, según esta lógica, cuando un partido se resuelve con una diferencia de goles igual o superior a 4,
el equipo perdedor recibirá una puntuación negativa de -1. \underline{Esto quiere decir que puede terminar la temporada con puntuación negativa.}\\\\
Ahora echemos un vistazo al módulo de Odoo para ver si se han efectuado correctamente las modificaciones. Sobra decir que es
necesario reiniciar el contenedor para que se apliquen los cambios.
\newpage
\begin{figure}[hbt!]
    \centering
    \includegraphics[width=1\textwidth]{12.png}
    \caption{Crearemos nuevos partidos de prueba}
\end{figure}
\begin{figure}[hbt!]
    \centering
    \includegraphics[width=1\textwidth]{13.png}
    \caption{Comprobamos que el sistema de puntuación se aplica correctamente}
\end{figure}
\newpage

\subsection{Botones para alterar los goles de los partidos}
\noindent \\01
\newpage

\subsection{Web Controller para eliminar empates}
\noindent \\01
\newpage

\subsection{Informe PDF por cada partido}
\noindent \\01
\newpage

\subsection{Wizard para crear nuevos partidos}
\noindent \\01
\newpage

\subsection{Vista Graph}
\noindent \\01
\newpage

\section{Actividad 02 – Pruebas \textit{EJ08-API-REST\_Socio}}
\noindent \\Para la realización de esta actividad he empleado las siguientes herramientas:\\
\begin{itemize}
    \item \textit{\textbf{Kdenlive:}} es un potente editor de video no lineal, libre y de código abierto, basado en el framework MLT. 
    Funciona en Linux, Windows y macOS, permitiendo la edición multipista de audio y video con múltiples efectos, transiciones y 
    herramientas de corrección de color, ideal tanto para usuarios principiantes como profesionales.
\end{itemize}
\begin{itemize}
    \item \textit{\textbf{Luvvoice:}} es una plataforma de inteligencia artificial diseñada para convertir texto en voz (Text-to-Speech) de forma
    realista y natural. Es una herramienta muy utilizada por creadores de contenido para narrar videos de YouTube o TikTok, así como
    por educadores para crear audiolibros o material didáctico.
\end{itemize}
\noindent \\El vídeo creado se divide en dos partes bien diferenciadas. En la primera parte se muestra el código de la API donde se puede
apreciar la url en la que realizar las peticiones y el formato en el que deben enviarse los datos. En la segunda parte se muestran las peticiones
http (POST, GET, PUT y DELETE) hechas con el programa \textit{\textbf{Postman}}. \\\\
La duración del vídeo no debe exceder de los 3 minutos (en este caso dura aproximadamente 2 minutos y medio). \\\\
El archivo \textit{Actividad 2.mp4} está ubicado en la carpeta \textit{practica5} del proyecto.
\newpage

\section{Actividad 03 – Bot Telegram API REST}
% \subsection{Informe PDF por cada partido}
% \noindent \\01
% \newpage
\noindent \\03
\newpage

\section{Actividad 04 – Generación imágenes aleatorias Web Controller}
% \subsection{Informe PDF por cada partido}
% \noindent \\01
% \newpage
\noindent \\04
\newpage

\end{document}