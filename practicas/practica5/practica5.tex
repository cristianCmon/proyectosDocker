% [Tamaño principal de la fuente del documento, tamaño del papel, con título (crea salto de página)]{Tipo de documento}
\documentclass[12pt, a4paper, titlepage]{article}
\usepackage[utf8]{inputenc}
% Traduce expresiones al español
\usepackage[spanish]{babel}
\usepackage{setspace}

\usepackage{graphicx}
\usepackage{geometry}
\geometry{margin=2.5cm}
\graphicspath{ {./capturas/} }

% Permite utilizar labeling para listar de forma personalizada
\usepackage{scrextend}
% Permite centrar verticalmente m{}
\usepackage{array}
% Permite colorear texto
\usepackage{xcolor}

\title{\LARGE \textbf{Práctica 5: Módulos Odoo Controlador, Herencia y Web Controllers.} \\[2ex] \Large Sistemas de Gestión Empresarial}
\author{\\[20ex]Cristian Fernández}
\date{\today}

\begin{document}
\maketitle

\doublespacing
\noindent \\
\tableofcontents % Índice

\newpage

\section{Introducción}
\noindent \\El objetivo de esta práctica es aplicar los conocimientos adquiridos en ejercicios anteriores sobre 
modelos, relaciones y vistas en el desarrollo de aplicaciones. Para ello, nos ocuparemos de la implementación 
de cuatro actividades que abarcan diferentes contextos, como la gestión de tareas, bibliotecas de cómics, 
pacientes y médicos, y ciclos formativos. \\

\noindent Habrá que tener en cuenta la revisión de los ejemplos proporcionados en el siguiente repositorio. \textit{https://github.com/sergarb1/OdooModulosEjemplos} \\

\noindent Una vez que verifiquemos y probemos el funcionamiento de cada ejemplo en nuestro contenedor de Odoo debemos:\\
\begin{itemize}
    \item Revisar los manifests.
    \item Revisar los modelos de datos.
    \item Revisar las vistas.
\end{itemize}

\noindent \\Entender y probar estos ejemplos nos facilitará la comprensión sobre la arquitectura de Odoo y su sistema de módular
y, por tanto, la realización la práctica.

\noindent \\Se reutilizará el repositorio creado en las prácticas anteriores, con objeto de tener todo el trabajo bien organizado y disponible
para una posible visualización en cualquier momento. \\
\noindent \\Cada actividad tendrá una explicación breve del proceso de creación y se demostrará su funcionamiento mediante capturas.
Si fuese necesario también se comentarán los problemas encontrados y las soluciones empleadas.
\newpage

\section{Actividad 01 – Modificación EJ07-LigaFutbol}
\noindent \\01
\newpage

\section{Actividad 02 – Pruebas `EJ08-API-REST\_Socio'}
\noindent \\02
\newpage

\section{Actividad 03 – Bot Telegram API REST}
\noindent \\03
\newpage

\section{Actividad 04 – Generación imágenes aleatorias Web Controller}
\noindent \\04
\newpage

\end{document}