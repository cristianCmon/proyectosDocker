% [Tamaño principal de la fuente del documento, tamaño del papel, con título (crea salto de página)]{Tipo de documento}
\documentclass[12pt, a4paper, titlepage]{article}
\usepackage[utf8]{inputenc}
% Traduce expresiones al español
\usepackage[spanish]{babel}
\usepackage{setspace}

\usepackage{graphicx}
\usepackage{geometry}
\geometry{margin=2.5cm}
\graphicspath{ {./capturas/} }

% Permite utilizar labeling para listar de forma personalizada
\usepackage{scrextend}
% Permite centrar verticalmente m{}
\usepackage{array}
% Permite colorear texto
\usepackage{xcolor}

\title{\LARGE \textbf{Práctica 4 - Desarrollo de módulos de Odoo: Modelo y Vista} \\[2ex] \Large Sistemas de Gestión Empresarial}
\author{\\[20ex]Cristian Fernández}
\date{\today}

\begin{document}
\maketitle

\doublespacing
\noindent \\
\tableofcontents % Índice

\newpage

\section{Introducción}
\noindent \\El objetivo de esta práctica es aplicar los conocimientos adquiridos en ejercicios anteriores sobre 
modelos, relaciones y vistas en el desarrollo de aplicaciones. Para ello, nos ocuparemos de la implementación 
de cuatro actividades que abarcan diferentes contextos, como la gestión de tareas, bibliotecas de cómics, 
pacientes y médicos, y ciclos formativos. \\

\noindent Habrá que tener en cuenta la revisión de los ejemplos proporcionados en el siguiente repositorio. \textit{https://github.com/sergarb1/OdooModulosEjemplos} \\

\noindent Una vez que verifiquemos y probemos el funcionamiento de cada ejemplo en nuestro contenedor de Odoo debemos:
\begin{itemize}
    \item Revisar los manifests.
    \item Revisar los modelos de datos.
    \item Revisar las vistas.
\end{itemize}

\noindent \\Entender y probar estos ejemplos nos facilitará la comprensión sobre la arquitectura de Odoo y su sistema de módular
y, por tanto, la realización la práctica.

\noindent \\Se reutilizará el repositorio creado en las prácticas anteriores, con objeto de tener todo el trabajo bien organizado y disponible
para una posible visualización en cualquier momento. \\
\noindent \\Cada actividad tendrá una explicación breve del proceso de creación y se demostrará su funcionamiento mediante capturas.
Si fuese necesario también se comentarán los problemas encontrados y las soluciones empleadas.
\newpage

\section{Actividad 01 - Lista de tareas}
\newpage

\section{Actividad 02 - Biblioteca de cómics}
\newpage

\section{Actividad 03 - Hospital}
\newpage

\section{Actividad 04 - Instituto}
\newpage
% \section{Implantación de módulo básico \textit{hola mundo}}
% \noindent \\Odoo permite la creación de módulos personalizados al gusto del usuario en los que puede extender funcionalidades según se requiera.
% Para demostrarlo crearemos un módulo de ejemplo básico muy sencillo cuya única finalidad será aparecer listado en la lista de módulos de nuestro
% contenedor Odoo. El propósito es que el sistema detecte y muestre correctamente el módulo, lo cual nos permitirá avanzar al siguiente punto.\\

% \noindent Asentada la premisa comenzaremos la tarea creando una carpeta dentro de nuestro proyecto llamada \textit{Ejemplo01-HolaMundo}, en el interior
% del directorio en el que hayamos configurado el alojamiento de nuestros módulos personalizados (en mi caso la carpeta \textit{addons}).\\

% \noindent La carpeta recién creada contendrá los siguientes ficheros:
% \begin{itemize}
%     % Es necesario escapar los guiones bajos para que no se produzcan errores
%     \item \textbf{Fichero \textit{\_\_init\_\_.py}:} Este fichero estará vacío. 
%     \item \textbf{Fichero \textit{\_\_manifest\_\_.py}:} Este fichero contendrá una línea de código (se mostrará en la siguiente captura).
% \end{itemize}

% \noindent \\Posteriormente levantaremos nuestros contenedores con docker compose, accederemos a Odoo en el navegador y buscaremos el módulo recién creado.\\

% \noindent Vamos a ver las capturas de todo el proceso en la siguiente página. 
% \newpage

% \begin{figure}[hbt!]
%     \centering
%     \includegraphics[width=1\textwidth]{ScreenShot_1.png}
%     \caption{Carpeta y archivos creados junto con la línea de código}
% \end{figure}
% \begin{figure}[hbt!]
%     \centering
%     \includegraphics[width=1\textwidth]{ScreenShot_2.png}
%     \caption{Dentro de Odoo pulsamos en \textit{Actualizar lista de aplicaciones}}
% \end{figure}
% \begin{figure}[hbt!]
%     \centering
%     \includegraphics[width=1\textwidth]{ScreenShot_3.png}
%     \caption{Nos saldrá este modal. Pulsamos en \textit{Actualizar}}
% \end{figure}
% \newpage

% \begin{figure}[hbt!]
%     \centering
%     \includegraphics[width=1\textwidth]{ScreenShot_4.png}
%     \caption{Eliminamos los filtros y buscamos "hola mundo". Si se visualiza nuestro módulo es que hemos tenido éxito y podemos seguir avanzando 
%     hasta el siguiente punto}
% \end{figure}

% \noindent \\Partiendo de la base de que nuestro proyecto está perfectamente configurado gracias a las prácticas anteriores, el único problema con
% el que me he topado es la necesidad de que nuestro Odoo debe estar configurado en modo desarrollador y para ello tenemos que hacer una pequeña configuración previa.
% \begin{figure}[hbt!]
%     \centering
%     \includegraphics[width=0.8\textwidth]{ScreenShot_5.png}
%     \caption{Menú Ajustes - \textit{Activar modo de desarrollador} y pulsar en \textit{Guardar}}
% \end{figure}
% \newpage

% \section{Implantación de primer módulo \textit{lista de tareas}}
% \noindent \\El siguiente paso consistirá en crear un tipo de módulo fácil de entender: uno que cree nuevos modelos de datos (ficheros maestros) y
% permita que se observen estos modelos a través de un nuevo menú.

% \noindent \\Pero antes una reflexión.

% \noindent \\Crear la estructura de un nuevo módulo de Odoo puede resultar engorroso y tedioso si se hace con cierta frecuencia (y muy probablemente en el contexto de un 
% desarrollador así sea) por lo que usaremos una funcionalidad que incorpora el propio Odoo llamada \underline{\textbf{scaffold}}, que creará automáticamente toda la
% estructura de un nuevo módulo consiguiendo entre otras cosas aliviar el tedio de tener que hacerlo manualmente y mejorando nuestra calidad de vida como
% desarrolladores (quizás esté exagerando pero ciertamente se agradece). \\

% \noindent Desde un terminal accederemos al interior de nuestro contenedor de Odoo con el comando ``\textit{\textbf{docker compose exec} {\color{blue}nombreServicio} \textbf{/bin/bash}}''
% y una vez dentro introducimos el comando ``\textit{\textbf{odoo scaffold} {\color{blue}nombreNuevoModulo /ruta/volumen-addons/}}''.

% \noindent \\Deberemos sustituir la parte coloreada en azul por los valores correctos según nuestra configuración, tal como se puede observar en la siguiente captura:
% \begin{figure}[hbt!]
%     \centering
%     \includegraphics[width=1\textwidth]{ScreenShot_6.png}
% \end{figure}
% \newpage

% \noindent Si todo ha ido bien se deberá crear el módulo con toda su estructura tal que así:
% \begin{figure}[hbt!]
%     \centering
%     \includegraphics[width=0.4\textwidth]{ScreenShot_7.png}
% \end{figure}

% \noindent \\Para evitar futuros problemas le daremos permisos de escritura al módulo dentro del contenedor con el comando ``\textit{chmod 777 -R /mnt/extra-addons/{\color{blue}nombreNuevoModulo}}''
% tal como muestra la captura:
% \begin{figure}[hbt!]
%     \centering
%     \includegraphics[width=0.8\textwidth]{ScreenShot_8.png}
% \end{figure}
% \newpage

% \noindent Ahora podemos comprobar que Odoo reconoce y muestra el módulo así que, como en el apartado anterior, actualizamos y búscamos el módulo nuevo:
% \begin{figure}[hbt!]
%     \centering
%     \includegraphics[width=1\textwidth]{ScreenShot_10.png}
% \end{figure}

% \noindent \\Y justo aquí nos topamos con el primer problema ya que, si pulsamos en \textit{Activar} nos encontraremos con el siguiente error:
% \begin{figure}[hbt!]
%     \centering
%     \includegraphics[width=1\textwidth]{ScreenShot_11.png}
% \end{figure}
% \newpage

% \noindent Por su forma de funcionar intermanente Odoo necesita reiniciar el servicio y actualizar el módulo para que se actualicen los cambios realizados en el mismo,
% así que vamos a arreglar el error anterior \underline{\textbf{descomentando todo el código que contengan los archivos}} y modificando el \textbf{\textit{\_\_manifest\_\_.py}},
% reiniciando el servicio y activando/actualizando el contenedor.\\

% \noindent \textbf{\textit{\_\_manifest\_\_.py}} quedaría tal que así:
% \begin{figure}[hbt!]
%     \centering
%     \includegraphics[width=0.95\textwidth]{ScreenShot_9.png}
%     \caption{Cambiamos nombre y descomentamos la línea del csv}
% \end{figure}
% \newpage

% \noindent Y, desde un terminal, con docker compose reiniciaremos el contenedor con el comando ``\textit{\textbf{docker compose restart} {\color{blue}nombreServicio}}''
% \begin{figure}[hbt!]
%     \centering
%     \includegraphics[width=1\textwidth]{ScreenShot_12.png}
% \end{figure}

% \noindent \\Si volvemos a intentar activar el contenedor esta vez sí procederá correctamente y nos mostrará este resultado:
% \begin{figure}[hbt!]
%     \centering
%     \includegraphics{ScreenShot_13.png}
% \end{figure}

% \noindent \\También nos dejará actualizar el módulo.
% \begin{figure}[hbt!]
%     \centering
%     \includegraphics{ScreenShot_14.png}
% \end{figure}
% \newpage

% \section{Modificación del primer módulo \textit{lista de tareas}}
% \noindent \\Vamos a modificar nuestro módulo de forma liviana para comprobar si se actualiza correctamente. Básicamente crearemos unos cuantos campos en el modelo 
% que sean descriptivos y los añadiremos a la vista para poder visualizarlos correctamente desde Odoo. \\
% Las siguientes capturas mostrarán la creación de una nueva tarea desde nuestro módulo en la que se visualizará claramente el proceso de creación y el resultado y,
% posteriormente, mostraré unas capturas con las modificaciones que he hecho en el código. \\ \\
% Desde la vista principal del módulo clicamos en el botón \textit{New}.
% \begin{figure}[hbt!]
%     \centering
%     \includegraphics[width=0.8\textwidth]{ScreenShot_15.png}
%     \caption{Ejemplo de una tarea nueva}
% \end{figure}
% \newpage

% \begin{figure}[hbt!]
%     \centering
%     \includegraphics[width=0.8\textwidth]{ScreenShot_16.png}
%     \caption{Una vez creada podremos visualizarla en la vista principal}
% \end{figure}
% \indent \\
% \begin{figure}[hbt!]
%     \centering
%     \includegraphics[width=0.8\textwidth]{ScreenShot_17.png}
%     \caption{Si hacemos clic en ella podremos visualizarla con más detalle}
% \end{figure}
% \noindent \\ \\Cuando todo está configurado y preparado resulta muy simple utilizar el módulo. Al final comentaré una problemática muy incómoda que me he encontrado.\\
% \\Pero antes las capturas del código en la siguiente página:
% \newpage

% \noindent \\
% \begin{figure}[hbt!]
%     \centering
%     \includegraphics{ScreenShot_18.png}
%     \caption{Captura del modelo}
% \end{figure}
% \newpage

% \begin{figure}[hbt!]
%     \centering
%     \includegraphics[width=1\textwidth]{ScreenShot_19.png}
%     \caption{Captura de la vista}
% \end{figure}
% \newpage

% \noindent \\En esta ocasión me he topado con el problema más tedioso y farragoso de la práctica. Y es que Odoo da muchos problemas a la hora de actualizar los módulos.
% En primer lugar, el modo desarrollador no se queda habilitado entre sesiones por lo que hay que activarlo manualmente cada vez que levantemos el contenedor o iniciemos sesión.\\
% \\Por si fuera poco entre accesos al módulo puede deshabilitarse el modo desarrollador sin venir a cuento teniendo que volver a activarlo manualmente lo cual es muy molesto.\\ \\
% La solución que empleé para trabajar con comodidad ha sido, una vez se activa el modo desarrollador, convertirse en superusuario tal como se ve a continuación.\\
% \begin{figure}[hbt!]
%     \centering
%     \includegraphics[width=1\textwidth]{ScreenShot_20.png}
% \end{figure}
% \newpage

\end{document}